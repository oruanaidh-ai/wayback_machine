\chapter{BSS}


\section{Mixing and demixing}

The two sources are mixed as follows:
\begin{align}
  y_1(t) & = s_1(t) + s_2(t) \\
  y_2(t) & = s_1(t+\delta_1) + s_2(t+\delta_2)
\end{align}
where $\delta_1$ and $\delta_2$ are delays that depend on the distance
between the microphones and the directions of the sources.


In the frequency domain:
\begin{align}
  Y_1(\omega) & = S_1(\omega) + S_2(\omega) \\
  Y_2(\omega) & = e^{-\omega\delta_1}S_1(\omega) + e^{-\omega\delta_2}S_2(\omega)
\end{align}

This is a scalar mixture! So we can just do PCA and we are done. Not
so fast. There is a permutation problem. There is no guarantee that
signal 1 will appear on ouput channel 1, or signal 2 on output channel
2. If we were to apply PCA to all the different frequencies then we
would still have the problem of determining which output should be on
which channel. If there are $N$ different frequencies then there will
be $2^N$ different possibilities to choose from.

In radar, there is only one frequency so this doesn't matter so we can
use PCA. Speech is broadband. We need a different approach.

Let us write the delay operation as follows:
\begin{align}
  z_1 & =  e^{-\omega\delta_1}\\
  z_2 & =  e^{-\omega\delta_2}
\end{align}

We can write the mixing process as:
\begin{equation}
  \begin{bmatrix}
    Y_1(\omega)\\
    Y_2(\omega)
  \end{bmatrix}
  =
  \begin{bmatrix}
  1 & 1\\
  z_1 & z_2
  \end{bmatrix}
  \begin{bmatrix}
  S_1(\omega)\\
  S_2(\omega)
    \end{bmatrix}
\end{equation}

The inverse is
\begin{equation}
  \begin{bmatrix}
    S_1(\omega)\\
    S_2(\omega)
  \end{bmatrix}
  =
  \frac{1}{z_2-z_1}
  \begin{bmatrix}
  z_2 & -1\\
  -z_1 & 1
  \end{bmatrix}
  \begin{bmatrix}
  Y_1(\omega)\\
  Y_2(\omega)
    \end{bmatrix}
 \end{equation}   

We are going to ignore the determinant term because of the possibility of dividing by zero (or by some number close to zero).

\begin{equation}
  \begin{bmatrix}
    X_1(\omega)\\
    X_2(\omega)
  \end{bmatrix}
  =
  \begin{bmatrix}
  z_2 & -1\\
  -z_1 & 1
  \end{bmatrix}
  \begin{bmatrix}
  Y_1(\omega)\\
  Y_2(\omega)
    \end{bmatrix}
 \end{equation}   


So, if we could just find $\delta_1$ and $\delta_2$, we would demix as follows:
\begin{align}
  x_1(t) & = y_1(t+\delta_2) - y_2(t) \\
  x_2(t) & = -y_1(t+\delta_1) + y_2(t)
\end{align}


\section{The cost function}

We know that $x_1(t)$ and $x_2(t)$ are uncorrelated. That means that
we should find $\delta_1$ and $\delta_2$ that minimize the correlation
between $x_1(t)$ and $x_2(t)$.

Jumping back into the frequency domain, we need to minimize the
magnitude of $X_1(\omega) \overline{X_2(\omega)}$. We take the complex
conjugate because this is not a convolution - it's a correlation:
\begin{align}
  X_1(\omega) \overline{X_2(\omega)} & = \left(z_2 Y_1(\omega) - Y_2(\omega)\right) \overline{\left(-z_1 Y_1(\omega) + Y_2(\omega)\right) }\\
  & = -z_2 \overline{z_1} Y_1 \overline{Y_1} + z_2 Y_1 \overline{Y_2} + \overline{z_1} Y_2 \overline{Y_1} - Y_2 \overline{Y_2}
\end{align}

We sum the square of the magnitude of $X_1(\omega)
\overline{X_2(\omega)}$ over all frequencies and call that the cost
function.  Summing the square of the magnitude has another
advantage. According to Parseval's Theorem, the cost function is
identical to the sum of the squares of the cross correlations between
$x_1(t)$ and $x_2(t)$ in the time domain. This is a nice result
because the time domain justification for this cost function is easy
to understand.


\subsection{Cost function details}

As always, the devil is in the details:

\begin{itemize}

\item The cost function is multimodal. That's a bad thing if we are
  using a standard optimization routine because, depending on starting
  position, it might not find the best solution at the global
  minimum. Beware the local minimum. To mitigate against this, we do a coarse
  grid search first. And then use an optimizer to clean up the answer.


\item We take the logarithm of the sum of squares. This is done for the sake of numerical stability because the cost function changes very quickly.

  \item Note that only $z_1$ and $z_2$ are functions of $\delta_1$ and $\delta_2$. We can precompute auto- and cross-correlations of  $Y_1 \overline{Y_1}$, $Y_1 \overline{Y_2}$, $Y_2 \overline{Y_1}$, and $Y_2 \overline{Y_2}$ in advance to save computation.


\end{itemize}



